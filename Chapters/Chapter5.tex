% Chapter Template

\chapter{Conclusiones} % Main chapter title
En este capítulo se presentan los resultados obtenidos sobre el trabajo realizado, las herramientas aprendidas durante el transcurso de la carrera que fueron fundamentales para la ejecución y se proponen ideas que permitan mejorar lo desarrollado.
\label{Chapter5} % Change X to a consecutive number; for referencing this chapter elsewhere, use \ref{ChapterX}

  
%----------------------------------------------------------------------------------------

%----------------------------------------------------------------------------------------
%	SECTION 1
%----------------------------------------------------------------------------------------
 
\section{Resultados obtenidos}
En el trabajo realizado se logró implementar de forma exitosa un prototipo correspondiente a un sistema de monitoreo de cultivos agrícolas. A través de las pruebas realizadas se pudo validar y demostrar su correcto funcionamiento.
Se pueden nombrar los siguientes logros obtenidos:
\begin{itemize}
\item Se fabricó un prototipo funcional y se lo instaló en un cultivo agrícola para la realización de pruebas integrales del sistema, garantizando cumplir todos los requerimientos funcionales.
\item Se diseñó el esquemático y el PCB de la tarjeta electrónica para el prototipo.
\item Se desarrolló el firmware sobre un sistema operativo de tiempo real, para optimizar los recursos del microcontrolador y hacer un código más modular y escalable.
\item Se configuraron paneles de visualización en ThingsBoard. Permitiendo visualizar las variables ambientales adquiridas por el prototipo instalado en el  cultivo agrícola.
\item Se implementó el envío de alarmas del sistema al usuario mediante SMS.
\item Se utilizó Git para control de versiones, ayudó a tener la trazabilidad y flexibilidad necesaria para llevar a cabo el desarrollo de manera ordenada.
\end{itemize}

Los requerimientos del trabajo fueron cubiertos de acuerdo con la planificación, con la siguiente modificación:

\begin{itemize}
    \item Se cambió el requerimiento de utilizar HTTP como protocolo de envío de datos a la nube. Se utilizó MQTT porque es un protocolo más rápido, liviano y utiliza el modelo publicador/suscriptor.
\end{itemize}

En el desarrollo del trabajo se aplicaron muchos de los conocimientos adquiridos a lo largo de la Carrera de Especialización en Sistemas Embebidos. Entre los conocimientos aplicados, se pueden destacar los adquiridos en las siguientes asignaturas:

\begin{itemize}
    \item Programación de Microcontroladores: se utilizaron conceptos básicos de programación, conceptos de modularización y uso de patrones de diseño. 
    \item Protocolos de Comunicación en Sistemas Embebidos: se utilizaron protocolos de comunicación I2C, UART y MQTT.
    \item Sistemas Operativos de Tiempo Real: se implementó el firmware sobre freeRTOS. Se utilizaron colas y semáforos, para la comunicación y sincronización entre tareas.    
    \item Testing de Software en Sistemas Embebidos: uso de TDD para el desarrollo de los drivers. Se utilizó Ceedling para el desarrollo de pruebas automáticas.
    \item Diseño de Circuitos Impresos: se utilizaron buenas prácticas de diseño electrónico para desarrollar el esquemático y el circuito impreso del prototipo. Se utilizó KICAD como herramienta de diseño electrónico.
\end{itemize}

%----------------------------------------------------------------------------------------
%	SECTION 2
%----------------------------------------------------------------------------------------
\section{Próximos pasos}

Si bien se lograron obtener los resultados esperados, a futuro se puede continuar
con el desarrollo en varias medidas. A continuación, se detallan los aspectos que sería conveniente tomar en consideración:
\begin{itemize}
    \item Realizar un nuevo diseño del hardware que integre a todo el sistema y no utilice módulos por separado.
    \item Implementar un mecanismo de actualización de firmware remoto.
    \item Incluir soporte para trabajar con energías renovables.
    \item Aumentar la seguridad al enviar los datos al servidor utilizando SSL.
\end{itemize}